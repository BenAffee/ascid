\def\errnumber{1}%номер сообщения
\def\errtype{INFO}%тип сообщения
\def\errmessage{сработал путь auth}%текст сообщения
\def\errrequest{/auth}%при каком запросе возникает сообщение
\def\errtext{Сообщение возникает при нажатии пользователем кнопки ОК в форме ввода имени пользователя и пароля. В поле err записывается имя пользователя, введённое в форму.}
%определение для вывода таблички с описанием ошибки
\def\geterr{\section{Сообщение \textnumero\errnumber}
	\begin{tabular}{|p{3cm}||p{4cm}||p{4cm}||p{4cm}|}
		
		\hline 
		\begin{center}\textbf{Номер сообщения}\end{center}&\begin{center}\textbf{Тип сообщения}\end{center} &\begin{center}\textbf{Текст сообщения}\end{center} & \begin{center}\textbf{Запрос}\end{center} \\
		
		\begin{center}\errnumber\end{center}&\begin{center}\errtype\end{center}
		&\begin{center}\errmessage\end{center} & \begin{center}\errrequest\end{center} \\
		
		\hline
		\multicolumn{4}{|p{16.5cm}|}{\errtext} \\
		\hline
		
\end{tabular} }

\geterr


\def\errnumber{2}%номер сообщения
\def\errtype{WARNING}%тип сообщения
\def\errmessage{имя пользователя не соответствует регулярному выражению}%текст сообщения
\def\errrequest{/auth}%при каком запросе возникает сообщение
\def\errtext{Сообщение возникает при получении формы входа пользователя, в случае если имя пользователя содержит символы кроме латиницы, цифр и знака подчёркивания. В поле err записывается имя пользователя, введённое в форму.}
\geterr