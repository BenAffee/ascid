\documentclass[a4paper,12pt]{extreport}

%поля
\hoffset=-17mm %отступ до блока текста слева
\voffset=-30mm %отступ до блока текста сверху
\textwidth=170mm %ширина блока текста
\textheight=265mm %высота блока текста

%пакеты для поддержки русского языка, переносов, шрифтов и тп
\usepackage[T2A]{fontenc}
\usepackage[utf8]{inputenc}
\usepackage[english,russian]{babel}
\usepackage{cmap}
\usepackage{pscyr}
%\usepackage{comment}

\usepackage{tabularx}

%шрифт TimesNewRoman
%\renewcommand{\rmdefault}{ftm}

%абзацный отступ
\setlength\parindent{5ex}

%переопределение стандартных заголовков
\addto{\captionsrussian}{\renewcommand*{\contentsname}{СОДЕРЖАНИЕ}}
\addto{\captionsrussian}{\renewcommand*{\bibname}{СПИСОК ЛИТЕРАТУРЫ}}
\addto{\captionsrussian}{\renewcommand*{\chaptername}{Раздел}}

%первая строка в первом абзаце раздела с абзацным отступом
\usepackage{indentfirst}

%настройка заголовков
\usepackage{titlesec}
\titleformat{\chapter}{\filcenter\large\bfseries}{\chaptername~ \thechapter.}{8pt}{\bfseries}{}
\titleformat{\section}{\normalsize}{\thesection}{1em}{}
\titleformat{\subsection}{\normalsize}{\thesubsection}{1em}{}

% Настройка вертикальных и горизонтальных отступов
\titlespacing*{\chapter}{0pt}{-30pt}{*2}
\titlespacing*{\section}{\parindent}{*4}{0pt}
\titlespacing*{\subsection}{\parindent}{*4}{*4}



%настройка оглавления
\usepackage{tocloft}
%отображать в оглавлении только главы и разделы
\setcounter{tocdepth}{2}
%настройка заголовка Оглавления
\setlength{\cftbeforetoctitleskip}{0pt}%отступ сверху
\setlength{\cftaftertoctitleskip}{0pt}%отступ снизу
\renewcommand{\cfttoctitlefont}{\Large\hspace{0.38\textwidth}\bfseries}%слово Оглавление по центру, заглавными буквами
%настройка заголовка разделов
\renewcommand{\cftchapfont}{\large\itshape\bfseries Раздел } %шрифт главы в оглавлении
\setlength{\cftsecindent}{0pt} %без отступа слева, для подзаголовка
\renewcommand{\cftsecnumwidth}{20pt} %без отступа слева, для подзаголовка
\renewcommand{\cftchapnumwidth}{20pt} %без отступа слева, для главы
\setlength{\cftchapindent}{40pt}
%настройка глав
\setlength{\cftbeforechapskip}{20pt}%отступ сверху, над главой
\setlength{\cftbeforesecskip}{10pt}%отступ сверху, над главой
%\renewcommand{\cftchapfont}{\normalsize}%главы не жирным шрифтом


%полуторный интервал
\renewcommand{\baselinestretch}{1.0}

%интервал между абзацами
\setlength{\parskip}{0pt}

%переопределяем колонтитулы
\usepackage{fancyhdr}
\pagestyle{fancy}
\fancyhead[LE,RO]{\bfseries \small \slshape \rightmark}
\fancyhead[LO,RE]{\bfseries \small \slshape \leftmark}

%с какой страницы стартовать счётчик страниц
\setcounter{page}{1}

%с какой страницы стартовать счётчик глав
\setcounter{chapter}{1}

\begin{document}
%=================================================================
%=============== СОДЕРЖАНИЕ ======================================
%=================================================================

\tableofcontents
\newpage

%=================================================================
%=============== ТЕКСТОВКА =======================================
%=================================================================
\chapter{Подготовка среды выполнения приложений NodeJS}
	\section{Общие сведения о среде выполнения}
	\section{Установка необходимых пакетов}
	\section{Настройка приложения для выполнения в режиме production}
	\section{Настройка приложения в окружении менеджера процессов pm2}
\chapter{Структура каталогов приложения}
	\section{Описание файлов, находящихся в <<корне>> приложения}
	\section{Каталог public}
	\section{Каталог views}
	\section{Каталог routes}

\chapter{Сообщения об ошибках и другие сообщения, порождаемые приложением}
\def\errnumber{1}%номер сообщения
\def\errtype{INFO}%тип сообщения
\def\errmessage{сработал путь auth}%текст сообщения
\def\errrequest{/auth}%при каком запросе возникает сообщение
\def\errtext{Сообщение возникает при нажатии пользователем кнопки ОК в форме ввода имени пользователя и пароля. В поле err записывается имя пользователя, введённое в форму.}
%определение для вывода таблички с описанием ошибки
\def\geterr{\section{Сообщение \textnumero\errnumber}
	\begin{tabular}{|p{3cm}||p{4cm}||p{4cm}||p{4cm}|}
		
		\hline 
		\begin{center}\textbf{Номер сообщения}\end{center}&\begin{center}\textbf{Тип сообщения}\end{center} &\begin{center}\textbf{Текст сообщения}\end{center} & \begin{center}\textbf{Запрос}\end{center} \\
		
		\begin{center}\errnumber\end{center}&\begin{center}\errtype\end{center}
		&\begin{center}\errmessage\end{center} & \begin{center}\errrequest\end{center} \\
		
		\hline
		\multicolumn{4}{|p{16.5cm}|}{\errtext} \\
		\hline
		
\end{tabular} }

\geterr


\def\errnumber{2}%номер сообщения
\def\errtype{WARNING}%тип сообщения
\def\errmessage{имя пользователя не соответствует регулярному выражению}%текст сообщения
\def\errrequest{/auth}%при каком запросе возникает сообщение
\def\errtext{Сообщение возникает при получении формы входа пользователя, в случае если имя пользователя содержит символы кроме латиницы, цифр и знака подчёркивания. В поле err записывается имя пользователя, введённое в форму.}
\geterr



\end{document}

