\def\errnumber{1}%номер сообщения
\def\errtype{INFO}%тип сообщения
\def\errmessage{сработал путь auth}%текст сообщения
\def\errrequest{/auth}%при каком запросе возникает сообщение
\def\errtext{Сообщение возникает при нажатии пользователем кнопки ОК в форме ввода имени пользователя и пароля. В поле err записывается имя пользователя, введённое в форму.}
%определение для вывода таблички с описанием ошибки
\def\geterr{\subsection{Сообщение \textnumero\errnumber}
	\noindent\begin{tabular}{|p{3cm}||p{4cm}||p{5cm}||p{4cm}|}
		
		\hline 
		\begin{center}\vspace{-0.9cm}\textbf{Номер сообщения}\end{center}&\begin{center}\vspace{-0.9cm}\textbf{Тип сообщения}\end{center} &\begin{center}\vspace{-0.9cm}\textbf{Текст сообщения}\end{center} & \begin{center}\vspace{-0.9cm}\textbf{Запрос}\end{center} \\
		
		\begin{center}\vspace{-0.9cm}\errnumber\end{center}&\begin{center}\vspace{-0.9cm}\errtype\end{center}
		&\begin{center}\vspace{-0.9cm}\errmessage\end{center} & \begin{center}\vspace{-0.9cm}\errrequest\end{center} \\
		
		\hline
		\multicolumn{4}{|p{16.5cm}|}{\errtext} \\
		\hline
		
\end{tabular} }

\geterr

%====================================================сообщение 2 ++=++++++++++++++++++++++++
\def\errnumber{2}%номер сообщения
\def\errtype{WARNING}%тип сообщения
\def\errmessage{имя пользователя не соответствует регулярному выражению}%текст сообщения
\def\errrequest{/auth}%при каком запросе возникает сообщение
\def\errtext{Сообщение возникает при получении формы входа пользователя, в случае если имя пользователя содержит символы кроме латиницы, цифр и знака подчёркивания. В поле err записывается имя пользователя, введённое в форму.}
\geterr

%====================================================сообщение 3 ++=++++++++++++++++++++++++
\def\errnumber{3}%номер сообщения
\def\errtype{INFO}%тип сообщения
\def\errmessage{вход в админ-панель}%текст сообщения
\def\errrequest{/control/1}%при каком запросе возникает сообщение
\def\errtext{Сообщение возникает при КОРРЕКТНОМ входе в админ-панель}
\geterr


%====================================================сообщение 4 ++=++++++++++++++++++++++++
\def\errnumber{4}%номер сообщения
\def\errtype{WARNING}%тип сообщения
\def\errmessage{Неудачная попытка входа в админ-панель}%текст сообщения
\def\errrequest{/control/1}%при каком запросе возникает сообщение
\def\errtext{Сообщение возникает при НЕУДАЧНОЙ попытке входа в админ-панель. Скорее всего у пользователя не установлен флаг isAdministrator. На это сообщение стоит обратить особое внимание, так как скорее всего это  попытка осознанного несанкционированного доступа}
\geterr

%====================================================сообщение 5 ++=++++++++++++++++++++++++
\def\errnumber{5}%номер сообщения
\def\errtype{ERROR}%тип сообщения
\def\errmessage{Ошибка получения списка ВСЕХ пользователей из базы}%текст сообщения
\def\errrequest{/control/1}%при каком запросе возникает сообщение
\def\errtext{Сообщение возникает при ошибке получения списка всех пользователей из базы при запросе типа find.  В поле err записывается json-тело ошибки}
\geterr

%====================================================сообщение 6 ++=++++++++++++++++++++++++
\def\errnumber{6}%номер сообщения
\def\errtype{WARNING}%тип сообщения
\def\errmessage{База вернула пустой результат при получении списка ВСЕХ пользователей}%текст сообщения
\def\errrequest{/control/1}%при каком запросе возникает сообщение
\def\errtext{Сообщение возникает при пустом ответе из базы, при получении списка всех пользователей, при запросе типа find.}
\geterr

%====================================================сообщение 7 ++=++++++++++++++++++++++++
\def\errnumber{7}%номер сообщения
\def\errtype{ERROR}%тип сообщения
\def\errmessage{Ошибка проверки формата даты на стороне сервера}%текст сообщения
\def\errrequest{noe\_functions.dateToMillis}%при каком запросе возникает сообщение
\def\errtext{Сообщение возникает в функции преобразования даты в миллисекундах переданной от пользователя. В поле err запишется значение принятое в качестве аргумента и путь в котором произошла ошибка}
\geterr


%====================================================сообщение 8 ++=++++++++++++++++++++++++
\def\errnumber{8}%номер сообщения
\def\errtype{INFO}%тип сообщения
\def\errmessage{удалён пользователь}%текст сообщения
\def\errrequest{admin/delete\_user}%при каком запросе возникает сообщение
\def\errtext{Сообщение возникает при успешном удалении пользователя. В поле err записывается какого пользователя удалили и кто удалил.}
\geterr

%====================================================сообщение 9 ++=++++++++++++++++++++++++
\def\errnumber{9}%номер сообщения
\def\errtype{WARNING}%тип сообщения
\def\errmessage{Неудачная попытка удаления пользователя}%текст сообщения
\def\errrequest{admin/delete\_user}%при каком запросе возникает сообщение
\def\errtext{Сообщение возникает при НЕУДАЧНОЙ попытке удаления пользователя. Скорее всего у пользователя не установлен флаг isAdministrator. На это сообщение стоит обратить особое внимание, так как скорее всего это  попытка осознанного несанкционированного доступа}
\geterr

%====================================================сообщение 10 ++=++++++++++++++++++++++++
\def\errnumber{10}%номер сообщения
\def\errtype{ERROR}%тип сообщения
\def\errmessage{ошибка удаления пользователя из базы}%текст сообщения
\def\errrequest{admin/delete\_user}%при каком запросе возникает сообщение
\def\errtext{Сообщение возникает при попытке удаления пользователя. При этом база вернула ошибку. В поле err пишется тело ошибки.}
\geterr


%====================================================сообщение 11 ++=++++++++++++++++++++++++
\def\errnumber{11}%номер сообщения
\def\errtype{WARNING}%тип сообщения
\def\errmessage{Неудачная попытка отобразить логи}%текст сообщения
\def\errrequest{admin/logs}%при каком запросе возникает сообщение
\def\errtext{Сообщение возникает при неудачной попытке удалить пользователя. Скорее всего у пользователя не установлен флаг isAdministrator. На это сообщение стоит обратить особое внимание, так как скорее всего это  попытка осознанного несанкционированного доступа.}
\geterr

%====================================================сообщение 12 ++=++++++++++++++++++++++++
\def\errnumber{12}%номер сообщения
\def\errtype{INFO}%тип сообщения
\def\errmessage{отображаем логи}%текст сообщения
\def\errrequest{admin/logs}%при каком запросе возникает сообщение
\def\errtext{Сообщение возникает при успешном отображении логов.}
\geterr


%====================================================сообщение 13 ++=++++++++++++++++++++++++
\def\errnumber{13}%номер сообщения
\def\errtype{ERROR}%тип сообщения
\def\errmessage{Ошибка получения списка ВСЕХ логов из базы}%текст сообщения
\def\errrequest{/admin/logs}%при каком запросе возникает сообщение
\def\errtext{Сообщение возникает при ошибке получения логов из базы при запросе типа find.  В поле err записывается json-тело ошибки}
\geterr


%====================================================сообщение 14 ++=++++++++++++++++++++++++
\def\errnumber{14}%номер сообщения
\def\errtype{WARNING}%тип сообщения
\def\errmessage{База вернула пустой результат при получении логов}%текст сообщения
\def\errrequest{/admin/logs}%при каком запросе возникает сообщение
\def\errtext{Сообщение возникает при пустом ответе из базы, при получении логов, при запросе типа find.}
\geterr

%====================================================сообщение 15 ++=++++++++++++++++++++++++
\def\errnumber{15}%номер сообщения
\def\errtype{ERROR}%тип сообщения
\def\errmessage{Ошибка получения списка ВСЕХ пользователей из базы при отображении логов}%текст сообщения
\def\errrequest{/logs}%при каком запросе возникает сообщение
\def\errtext{Сообщение возникает при ошибке получения списка всех пользователей из базы при отображении логов при запросе типа find.  В поле err записывается json-тело ошибки}
\geterr

%====================================================сообщение 16 ++=++++++++++++++++++++++++
\def\errnumber{16}%номер сообщения
\def\errtype{WARNING}%тип сообщения
\def\errmessage{База вернула пустой результат при получении списка ВСЕХ пользователей при отображении логов}%текст сообщения
\def\errrequest{/logs}%при каком запросе возникает сообщение
\def\errtext{Сообщение возникает при пустом ответе из базы, при получении списка всех пользователей, при отображении логов, при запросе типа find.}
\geterr

%====================================================сообщение 17 ++=++++++++++++++++++++++++
\def\errnumber{17}%номер сообщения
\def\errtype{ERROR}%тип сообщения
\def\errmessage{ошибка удаления информации о пользователе у подчинённых}%текст сообщения
\def\errrequest{/admin/delete\_user}%при каком запросе возникает сообщение
\def\errtext{Сообщение возникает при ошибках из базы при удалении записи с именем удаляемого пользователя у его подчинённых. В поле err -- тело ошибки}
\geterr

%====================================================сообщение 18 ++=++++++++++++++++++++++++
\def\errnumber{18}%номер сообщения
\def\errtype{ERROR}%тип сообщения
\def\errmessage{ошибка удаления информации о пользователе у командиров}%текст сообщения
\def\errrequest{/admin/delete\_user}%при каком запросе возникает сообщение
\def\errtext{Сообщение возникает при ошибках из базы при удалении записи с именем удаляемого пользователя у его командиров. В поле err -- тело ошибки}
\geterr